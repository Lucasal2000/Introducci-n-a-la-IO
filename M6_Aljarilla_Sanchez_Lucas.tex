\documentclass{article}
\usepackage{amssymb, amsmath}
\usepackage{graphicx}
\usepackage{hyperref}
\usepackage[spanish]{babel}
\hypersetup{
colorlinks=true,
citecolor=blue,
urlcolor=blue
}
\author{Lucas Aljarilla Sanchez}
\title{Introducción a la Investigación Operativa}
\begin{document}
\maketitle
\begin{abstract}
Breve introducción a la investigación operativa para aquelos que quieren saber un poco más de este mundillo. Veremos cómo surgió, los tipos que hay y sus usos actuales. Para más información sobre este trabajo mirar la url: \url{https://github.com/Lucasal2000/proyecto_final} 
\end{abstract}
 \vfill
\textbf{Palabras clave}
\begin{itemize}
\item  \underline{modelo matemático}: formula matemática empleada para expresar relaciones, proposiciones sustantivas de hechos, variables, parámetros, entidades y relaciones entre variables de las operaciones, para estudiar comportamientos de sistemas complejos ante situaciones difíciles de observar en la realidad.
\item  \underline{modelo determinístico}: modelo matemático donde las condiciones iniciales producen invariablemente los mismos resultados.
\item  \underline{modelo estocástico}: modelo matemático con un comportamiento no determinístico, por lo que su resultado es aleatorio.
\item  \underline{sistema}: un conjunto de elementos relacionados entre sí que funciona como un todo, en nuestro caso el sistema es el modelo con sus respectivas restricciones. 
\end{itemize}

\section{Introducción}

Muchos os preguntareis qué es eso de la investigación operativa. Bien, según Ackoff, Arnoff y Churchman: "La investigación operativa es la aplicación, por grupos interdisciplinarios, del método científico a problemas relacionados con el control de las organizaciones o sistemas (Hombre-Máquina) a fin de que se produzcan soluciones que mejor sirvan a los objetivos de toda la organización." \cite{Taha2004}

En resumidas cuentas, es la rama de las matemáticas que se ocupa de la toma de decisiones óptimas y de modelar sistemas determinísticos y estocásticos que se originan en la vida real.

\section{Etapas de un problema de investigación operativa} \cite{Luenberger2005}

\subsection{Formulación del problema}
Estudiar el sistema que se va a analizar:Definir el problema, especificar los objetivos y las limitaciones bajo las cuales opera el sistema que se modeliza.
\subsection{Construcción del modelo}
Modelo: Representación idealizada del sistema que reproduce la realidad de la forma más fiel posible, tratando de entender cómo se comporta el mundo real.
\subsection{Obtención de solución}
Obtener una solución óptima teniendo en cuenta que estas soluciones son óptimas sólo respecto al modelo utilizado.
\subsection{Validación del modelo}
Comprobar si el modelo propuesto hace lo que se supone que debe hacer.
¿Proporciona una predicción razonable del comportamiento del sistema que se está estudiando?
\subsection{Puesta en practica}
Poner en práctica la solución final.
\section{Modelos de investigación operativa}
\begin{minipage}[b]{0.5\linewidth}
\begin{itemize}
\item \textbf{Determinísticos}
\item Análisis de redes
\item Programación lineal multiobjetivo
\item Programación lineal
\end{itemize}
    \end{minipage} 
    \begin{minipage}[b]{0.5\linewidth} 
\begin{itemize}
\item \textbf{Estocásticos}
\item Teoria de intervalos
\item Teoría de juegos
\item Teoria de colas
\end{itemize}    
\end{minipage}

\section{Aplicaciones de la investigación operativa}
La investigación operativa es usada para un montón de cosas en nuestro día a día, pero unos de los ejemplos más famosos y más estudiados son: el problema de la dieta, el problema de transporte y el problema de cartera de valores.

\section{Programación lineal}
Dentro de la investigación operativa encontramos los problemas de programación lineal. \cite{Rios-InsuaS.MateosA.BielzaM.C.yJimenez2004}

Un problema de programación lineal es un programa matemático en el cual la función objetivo es lineal en las variables de decisión y cada restricción es una desigualdad lineal. Además tiene una restricción de signo; es decir, las variables de decisión son no negativas. Vamos a ver un pequeño ejemplo:

\textbf{Ejemplo} \cite{GobernaM.A.JornetV.Puente2004}
\begin{equation}
\begin{split}
MaxZ&=2x_1+2x_2\\
s.a.~&-x_1+x_2\leq 2\\
&x_1+2x_2\leq 6\\
&2x_1+x_2\leq 6\\
&x_1,x_2\geq 0
\end{split}
\end{equation}
\begin{center}
\includegraphics[width=.70\textwidth]{ej_grafi.png} \\
\end{center}

\begin{center}
\begin{tabular}{ c | c }
Soluciones ($x_1,x_2$)  & Valor de la función objetivo (Z) \\ \hline 
(0,0) & Z=0 \\
(0,2) & Z=2 \\
(2/3,8/3) & Z=4 \\
(2,2) & Z=6 \\
(3,0) & Z=6
\end{tabular}
\end{center}

\bibliographystyle{unsrt}
\bibliography{libreria}

\end{document}