\documentclass{article}
\usepackage[spanish]{babel}
\author{Lucas Aljarilla Sanchez}
\title{Introducción a la Investigación Operativa}
\begin{document}
\maketitle
\begin{abstract}
Breve introducción a la investigación operativa para aquelos que quieren saber un poco más de este mundillo. Veremos cómo surgió, los tipos que hay y sus usos actuales.
\end{abstract}
\section{Introducción}

Muchos os preguntareis qué es eso de la investigación operativa. Bien, según Ackoff, Arnoff y Churchman: "La investigación operativa es la aplicación, por grupos interdisciplinarios, del método científico a problemas relacionados con el control de las organizaciones o sistemas (Hombre-Máquina) a fin de que se produzcan soluciones que mejor sirvan a los objetivos de toda la organización."

En resumidas cuentas, es la rama de las matemáticas que se ocupa de la toma de decisiones óptimas y de modelar sistemas determinísticos y estocásticos que se originan en la vida real.

\section{Etapas de un problema de investigación operativa}

\subsection{Formulación del problema}
\subsection{Construcción del modelo}
\subsection{Obtención de solución}
\subsection{Validación del modelo}
\subsection{Puesta en practica}









\end{document}