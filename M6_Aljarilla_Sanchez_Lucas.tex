\documentclass{article}
\usepackage[spanish]{babel}
\usepackage{hyperref}
\hypersetup{
colorlinks=true,
citecolor=blue,
urlcolor=blue
}
\author{Lucas Aljarilla Sanchez}
\title{Introducción a la Investigación Operativa}
\begin{document}
\maketitle
\begin{abstract}
Breve introducción a la investigación operativa para aquelos que quieren saber un poco más de este mundillo. Veremos cómo surgió, los tipos que hay y sus usos actuales. Para más información sobre este trabajo mirar la url: \url{https://github.com/Lucasal2000/proyecto_final} 
\end{abstract}
\section{Introducción}

Muchos os preguntareis qué es eso de la investigación operativa. Bien, según Ackoff, Arnoff y Churchman: "La investigación operativa es la aplicación, por grupos interdisciplinarios, del método científico a problemas relacionados con el control de las organizaciones o sistemas (Hombre-Máquina) a fin de que se produzcan soluciones que mejor sirvan a los objetivos de toda la organización."

En resumidas cuentas, es la rama de las matemáticas que se ocupa de la toma de decisiones óptimas y de modelar sistemas determinísticos y estocásticos que se originan en la vida real.

\section{Etapas de un problema de investigación operativa}

\subsection{Formulación del problema}
Estudiar el sistema que se va a analizar:Definir el problema, especificar los objetivos y las limitaciones bajo las cuales opera el sistema que se modeliza.
\subsection{Construcción del modelo}
Modelo: Representación idealizada del sistema que reproduce la realidad de la forma más fiel posible, tratando de entender cómo se comporta el mundo real.
\subsection{Obtención de solución}
Obtener una solución óptima teniendo en cuenta que estas soluciones son óptimas sólo respecto al modelo utilizado.
\subsection{Validación del modelo}
Comprobar si el modelo propuesto hace lo que se supone que debe hacer.
¿Proporciona una predicción razonable del comportamiento del sistema que se está estudiando?
\subsection{Puesta en practica}
Poner en práctica la solución final.
\section{Modelos de investigación operativa}
\begin{minipage}[b]{0.5\linewidth}
\begin{itemize}
\item \textbf{Determinísticos}
\item Análisis de redes
\item Programación lineal multiobjetivo
\item Programación lineal
\end{itemize}
    \end{minipage} 
    \begin{minipage}[b]{0.5\linewidth} 
\begin{itemize}
\item \textbf{Estocásticos}
\item Teoria de intervalos
\item Teoría de juegos
\item Teoria de colas
\end{itemize}    
\end{minipage}

\section{Aplicaciones de la investigación operativa}
\subsection{Problema de transporte}
\subsection{Problema de planificación de producción}
\subsection{Problema de de la dieta}
\subsection{Problema de cartera de valores}
\end{document}